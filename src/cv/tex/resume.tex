%!TEX TS-program = xelatex
%!TEX encoding = UTF-8 Unicode
% Awesome CV LaTeX Template for CV/Resume
%
% This template has been downloaded from:
% https://github.com/posquit0/Awesome-CV
%
% Author:
% Claud D. Park <posquit0.bj@gmail.com>
% http://www.posquit0.com
%
% Template license:
% CC BY-SA 4.0 (https://creativecommons.org/licenses/by-sa/4.0/)
%


%-------------------------------------------------------------------------------
% CONFIGURATIONS
%-------------------------------------------------------------------------------
% A4 paper size by default, use 'letterpaper' for US letter
\documentclass[11pt, a4paper]{awesome-cv}

% Configure page margins with geometry
\geometry{left=1.4cm, top=.8cm, right=1.4cm, bottom=1.8cm, footskip=.5cm}

% Specify the location of the included fonts
\fontdir[fonts/]

% Color for highlights
% Awesome Colors: awesome-emerald, awesome-skyblue, awesome-red, awesome-pink, awesome-orange
%                 awesome-nephritis, awesome-concrete, awesome-darknight
\colorlet{awesome}{awesome-red}
% Uncomment if you would like to specify your own color
% \definecolor{awesome}{HTML}{CA63A8}

% Colors for text
% Uncomment if you would like to specify your own color
% \definecolor{darktext}{HTML}{414141}
% \definecolor{text}{HTML}{333333}
% \definecolor{graytext}{HTML}{5D5D5D}
% \definecolor{lighttext}{HTML}{999999}

% Set false if you don't want to highlight section with awesome color
\setbool{acvSectionColorHighlight}{true}

% If you would like to change the social information separator from a pipe (|) to something else
\renewcommand{\acvHeaderSocialSep}{\quad\textbar\quad}

% require luacode for lua coding
\usepackage{luacode}

\begin{document}
% read in JSON resume data
% build helper printer fns
% TODO: at some point refactor print calls to fn, not
%       sure why I can't get them working right now
\newcommand{\dataPath}{../../../dist/files}
\begin{luacode}
  require("lualibs.lua")
  local f = io.open('\dataPath/resume.json', 'r')
  local s = f:read('*a')
  f:close()
  resume =  utilities.json.tolua(s)
  contact = resume['contact']
\end{luacode}
%-------------------------------------------------------------------------------
%	PERSONAL INFORMATION
%	Comment any of the lines below if they are not required
%-------------------------------------------------------------------------------
% Available options: circle|rectangle,edge/noedge,left/right
\name
  {\directlua{tex.print(contact['firstName'])}}
  {\directlua{tex.print(contact['lastName'])}}
\position{\directlua{tex.print(contact['title'])}}
\address{\directlua{tex.print(contact['currentLocation'])}}
\mobile{(+82) 10-9030-1843}
\email{posquit0.bj@gmail.com}
\homepage{www.posquit0.com}
\github{posquit0}
\linkedin{posquit0}
% \photo[rectangle,edge,right]{./examples/profile}
% \gitlab{gitlab-id}
% \stackoverflow{SO-id}{SO-name}
% \twitter{@twit}
% \skype{skype-id}
% \reddit{reddit-id}
% \medium{madium-id}
% \googlescholar{googlescholar-id}{name-to-display}
%% \firstname and \lastname will be used
% \googlescholar{googlescholar-id}{}
% \extrainfo{extra informations}
% \quote{``Be the change that you want to see in the world."}

%-------------------------------------------------------------------------------
% Print the header with above personal informations
% Give optional argument to change alignment(C: center, L: left, R: right)

\makecvheader[C]

% Print the footer with 3 arguments(<left>, <center>, <right>)
% Leave any of these blank if they are not needed
\makecvfooter
  {\today}
  {Byungjin Park~~~·~~~Résumé}
  {\thepage}


%-------------------------------------------------------------------------------
%	CV/RESUME CONTENT
%	Each section is imported separately, open each file in turn to modify content
%-------------------------------------------------------------------------------
%-------------------------------------------------------------------------------
%	SECTION TITLE
%-------------------------------------------------------------------------------
\cvsection{Summary}

%-------------------------------------------------------------------------------
%	CONTENT
%-------------------------------------------------------------------------------
\begin{cvparagraph}

\directlua{tex.print(resume['contact']['summary'])}

\end{cvparagraph}
%---------------------------------------------------------

%-------------------------------------------------------------------------------
%	SECTION TITLE
%-------------------------------------------------------------------------------
\cvsection{Work Experience}

%-------------------------------------------------------------------------------
%	LUA
%-------------------------------------------------------------------------------

%-------------------------------------------------------------------------------
%	CONTENT
%-------------------------------------------------------------------------------
\begin{cventries}

%---------------------------------------------------------


%---------------------------------------------------------

\begin{luacode}
  for k,exp in ipairs(resume['experience']) do
    local items = '\\begin{cvitems}'
    for j,bullet in ipairs(exp['bulletPoints']) do
    items = items .. string.format('\\item {\%s}', bullet)
    end
    items = items .. '\\end{cvitems}'

    tex.sprint(
      string.format(
        '\\cventry{\%s}{\%s}{\%s}{\%s - \%s}{\%s}',
        exp['jobTitle'],
        exp['company'],
        exp['location'],
        exp['start'],
        exp['end'],
        items
      )
    )
    tex.sprint('\\vspace{3mm}')
  end
\end{luacode}

% \cventry
%   {Some Job} % Job title
%   {Omnious. Co., Ltd.} % Organization
%   {Seoul, S.Korea} % Location
%   {Jun. 2017 - May. 2018} % Date(s)
%   {
%     \begin{cvitems} % Description(s) of tasks/responsibilities
%       \item {Provisioned an easily managable hybrid infrastructure(Amazon AWS + On-premise) utilizing IaC(Infrastructure as Code) tools like Ansible, Packer and Terraform.}
%       \item {Built fully automated CI/CD pipelines on CircleCI for containerized applications using Docker, AWS ECR and Rancher.}
%       \item {Designed an overall service architecture and pipelines of the Machine Learning based Fashion Tagging API SaaS product with the micro-services architecture.}
%       \item {Implemented several API microservices in Node.js Koa and in the serverless AWS Lambda functions.}
%       \item {Deployed a centralized logging environment(ELK, Filebeat, CloudWatch, S3) which gather log data from docker containers and AWS resources.}
%       \item {Deployed a centralized monitoring environment(Grafana, InfluxDB, CollectD) which gather system metrics as well as docker run-time metrics.}
%     \end{cvitems}
%   }

%---------------------------------------------------------
\end{cventries}

\input{resume/honors.tex}
\input{resume/presentation.tex}
\input{resume/writing.tex}
\input{resume/committees.tex}
%-------------------------------------------------------------------------------
%	SECTION TITLE
%-------------------------------------------------------------------------------
\cvsection{Education}


%-------------------------------------------------------------------------------
%	CONTENT
%-------------------------------------------------------------------------------
\begin{cventries}

%---------------------------------------------------------

  \begin{luacode}
    for k,edu in ipairs(resume['education']) do
      tex.sprint(
        string.format(
          '\\cventry{\%s, \%s}{\%s}{\%s}{\%s - \%s}{}',
          edu['degrees'][1]['degreeName'],
          edu['degrees'][1]['focus'],
          edu['organization'],
          edu['location'],
          edu['start'],
          edu['end']
        )
      )
      tex.sprint('\\vspace{3mm}')
    end
  \end{luacode}

  % \cventry
  %   {B.S. in Computer Science and Engineering} % Degree
  %   {POSTECH(Pohang University of Science and Technology)} % Institution
  %   {Pohang, S.Korea} % Location
  %   {Mar. 2010 - Aug. 2017} % Date(s)
  %   {
  %     \begin{cvitems} % Description(s) bullet points
  %       \item {Got a Chun Shin-Il Scholarship which is given to promising students in CSE Dept.}
  %     \end{cvitems}
  %   }

%---------------------------------------------------------
\end{cventries}

\input{resume/extracurricular.tex}


%-------------------------------------------------------------------------------
\end{document}
