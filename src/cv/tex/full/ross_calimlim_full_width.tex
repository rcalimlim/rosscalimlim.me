%!TEX TS-program = xelatex
%!TEX encoding = UTF-8 Unicode
% Awesome CV LaTeX Template for CV/Resume
%
% This template has been downloaded from:
% https://github.com/posquit0/Awesome-CV
%
% Author:
% Claud D. Park <posquit0.bj@gmail.com>
% http://www.posquit0.com
%
% Template license:
% CC BY-SA 4.0 (https://creativecommons.org/licenses/by-sa/4.0/)
%


%-------------------------------------------------------------------------------
% CONFIGURATIONS
%-------------------------------------------------------------------------------
% A4 paper size by default, use 'letterpaper' for US letter
\documentclass[11pt, a4paper]{awesome-cv}

% Configure page margins with geometry
\geometry{left=1.4cm, top=.8cm, right=1.4cm, bottom=1.8cm, footskip=.5cm}

% Specify the location of the included fonts
\fontdir[fonts/]

% Color for highlights
% Awesome Colors: awesome-emerald, awesome-skyblue, awesome-red, awesome-pink, awesome-orange
%                 awesome-nephritis, awesome-concrete, awesome-darknight
\colorlet{awesome}{awesome-red}
% Uncomment if you would like to specify your own color
% \definecolor{awesome}{HTML}{CA63A8}

% Colors for text
% Uncomment if you would like to specify your own color
% \definecolor{darktext}{HTML}{414141}
% \definecolor{text}{HTML}{333333}
% \definecolor{graytext}{HTML}{5D5D5D}
% \definecolor{lighttext}{HTML}{999999}

% Set false if you don't want to highlight section with awesome color
\setbool{acvSectionColorHighlight}{true}

% If you would like to change the social information separator from a pipe (|) to something else
\renewcommand{\acvHeaderSocialSep}{\quad\textbar\quad}

% require luacode for lua coding
\usepackage{luacode}
\usepackage{pgffor}

\begin{document}
% read in JSON resume data
% build helper printer fns
% TODO: at some point refactor print calls to fn, not
%       sure why I can't get them working right now
\newcommand{\dataPath}{../../../../dist/files}
\begin{luacode}
  require("lualibs.lua")
  local f = io.open('\dataPath/resume.json', 'r')
  local s = f:read('*a')
  f:close()
  resume =  utilities.json.tolua(s)
  contact = resume['contact']
\end{luacode}
%-------------------------------------------------------------------------------
%	PERSONAL INFORMATION
%	Comment any of the lines below if they are not required
%-------------------------------------------------------------------------------
% Available options: circle|rectangle,edge/noedge,left/right
\name
  {\directlua{tex.print(contact['firstName'])}}
  {\directlua{tex.print(contact['lastName'])}}
\position{\directlua{tex.print(contact['title'])}}
\address{\directlua{tex.print(contact['currentLocation'])}}
\mobile{\directlua{tex.print(contact['phone']['displayText'])}}
\email{\directlua{tex.print(contact['email']['displayText'])}}
\homepage{\directlua{tex.print(contact['site']['displayText'])}}
\github{\directlua{tex.print(contact['github']['displayText'])}}
\linkedin{\directlua{tex.print(contact['linkedin']['displayText'])}}
% \photo[rectangle,edge,right]{./examples/profile}
% \gitlab{gitlab-id}
% \stackoverflow{SO-id}{SO-name}
% \twitter{@twit}
% \skype{skype-id}
% \reddit{reddit-id}
% \medium{madium-id}
% \googlescholar{googlescholar-id}{name-to-display}
%% \firstname and \lastname will be used
% \googlescholar{googlescholar-id}{}
% \extrainfo{extra informations}
% \quote{``Be the change that you want to see in the world."}

%-------------------------------------------------------------------------------
% Print the header with above personal informations
% Give optional argument to change alignment(C: center, L: left, R: right)

\makecvheader[L]

% Print the footer with 3 arguments(<left>, <center>, <right>)
% Leave any of these blank if they are not needed
% \makecvfooter{}{}{}

%-------------------------------------------------------------------------------
%	CV/RESUME CONTENT
%	Each section is imported separately, open each file in turn to modify content
%-------------------------------------------------------------------------------
%-------------------------------------------------------------------------------
%	SECTION TITLE
%-------------------------------------------------------------------------------
\cvsection{Summary}

%-------------------------------------------------------------------------------
%	CONTENT
%-------------------------------------------------------------------------------
\begin{cvparagraph}

\directlua{tex.print(resume['contact']['summary'])}

\end{cvparagraph}
%---------------------------------------------------------

%-------------------------------------------------------------------------------
%	SECTION TITLE
%-------------------------------------------------------------------------------
\cvsection{Experience}


%-------------------------------------------------------------------------------
%	CONTENT
%-------------------------------------------------------------------------------
\begin{cventries}

%---------------------------------------------------------
  \cventry
    {Software Architect} % Job title
    {Omnious. Co., Ltd.} % Organization
    {Seoul, S.Korea} % Location
    {Jun. 2017 - May. 2018} % Date(s)
    {
      \begin{cvitems} % Description(s) of tasks/responsibilities
        \item {Provisioned an easily managable hybrid infrastructure(Amazon AWS + On-premise) utilizing IaC(Infrastructure as Code) tools like Ansible, Packer and Terraform.}
        \item {Built fully automated CI/CD pipelines on CircleCI for containerized applications using Docker, AWS ECR and Rancher.}
        \item {Designed an overall service architecture and pipelines of the Machine Learning based Fashion Tagging API SaaS product with the micro-services architecture.}
        \item {Implemented several API microservices in Node.js Koa and in the serverless AWS Lambda functions.}
        \item {Deployed a centralized logging environment(ELK, Filebeat, CloudWatch, S3) which gather log data from docker containers and AWS resources.}
        \item {Deployed a centralized monitoring environment(Grafana, InfluxDB, CollectD) which gather system metrics as well as docker run-time metrics.}
      \end{cvitems}
    }

%---------------------------------------------------------
  \cventry
    {Co-founder \& Software Engineer} % Job title
    {PLAT Corp.} % Organization
    {Seoul, S.Korea} % Location
    {Jan. 2016 - Jun. 2017} % Date(s)
    {
      \begin{cvitems} % Description(s) of tasks/responsibilities
        \item {Implemented RESTful API server for car rental booking application(CARPLAT in Google Play).}
        \item {Built and deployed overall service infrastructure utilizing Docker container, CircleCI, and several AWS stack(Including EC2, ECS, Route 53, S3, CloudFront, RDS, ElastiCache, IAM), focusing on high-availability, fault tolerance, and auto-scaling.}
        \item {Developed an easy-to-use Payment module which connects to major PG(Payment Gateway) companies in Korea.}
      \end{cvitems}
    }

%---------------------------------------------------------
  \cventry
    {Researcher} % Job title
    {Undergraduate Research, Machine Learning Lab(Prof. Seungjin Choi)} % Organization
    {Pohang, S.Korea} % Location
    {Mar. 2016 - Exp. Jun. 2017} % Date(s)
    {
      \begin{cvitems} % Description(s) of tasks/responsibilities
        \item {Researched classification algorithms(SVM, CNN) to improve accuracy of human exercise recognition with wearable device.}
        \item {Developed two TIZEN applications to collect sample data set and to recognize user exercise on SAMSUNG Gear S.}
      \end{cvitems}
    }

%---------------------------------------------------------
  \cventry
    {Software Engineer \& Security Researcher (Compulsory Military Service)} % Job title
    {R.O.K Cyber Command, MND} % Organization
    {Seoul, S.Korea} % Location
    {Aug. 2014 - Apr. 2016} % Date(s)
    {
      \begin{cvitems} % Description(s) of tasks/responsibilities
        \item {Lead engineer on agent-less backtracking system that can discover client device's fingerprint(including public and private IP) independently of the Proxy, VPN and NAT.}
        \item {Implemented a distributed web stress test tool with high anonymity.}
        \item {Implemented a military cooperation system which is web based real time messenger in Scala on Lift.}
      \end{cvitems}
    }

%---------------------------------------------------------
  \cventry
    {Game Developer Intern at Global Internship Program} % Job title
    {NEXON} % Organization
    {Seoul, S.Korea \& LA, U.S.A} % Location
    {Jan. 2013 - Feb. 2013} % Date(s)
    {
      \begin{cvitems} % Description(s) of tasks/responsibilities
        \item {Developed in Cocos2d-x an action puzzle game(Dragon Buster) targeting U.S. market.}
        \item {Implemented API server which is communicating with game client and In-App Store, along with two other team members who wrote the game logic and designed game graphics.}
        \item {Won the 2nd prize in final evaluation.}
      \end{cvitems}
    }

%---------------------------------------------------------
  \cventry
    {Researcher for <Detecting video’s torrents using image similarity algorithms>} % Job title
    {Undergraduate Research, Computer Vision Lab(Prof. Bohyung Han)} % Organization
    {Pohang, S.Korea} % Location
    {Sep. 2012 - Feb. 2013} % Date(s)
    {
      \begin{cvitems} % Description(s) of tasks/responsibilities
        \item {Researched means of retrieving a corresponding video based on image contents using image similarity algorithm.}
        \item {Implemented prototype that users can obtain torrent magnet links of corresponding video relevant to an image on web site.}
      \end{cvitems}
    }

%---------------------------------------------------------
  \cventry
    {Software Engineer Trainee} % Job title
    {Software Maestro (funded by Korea Ministry of Knowledge and Economy)} % Organization
    {Seoul, S.Korea} % Location
    {Jul. 2012 - Jun. 2013} % Date(s)
    {
      \begin{cvitems} % Description(s) of tasks/responsibilities
        \item {Performed research memory management strategies of OS and implemented in Python an interactive simulator for Linux kernel memory management.}
      \end{cvitems}
    }

%---------------------------------------------------------
  \cventry
    {Software Engineer} % Job title
    {ShitOne Corp.} % Organization
    {Seoul, S.Korea} % Location
    {Dec. 2011 - Feb. 2012} % Date(s)
    {
      \begin{cvitems} % Description(s) of tasks/responsibilities
        \item {Developed a proxy drive smartphone application which connects proxy driver and customer. Implemented overall Android application logic and wrote API server for community service, along with lead engineer who designed bidding protocol on raw socket and implemented API server for bidding.}
      \end{cvitems}
    }

%---------------------------------------------------------
  \cventry
    {Freelance Penetration Tester} % Job title
    {SAMSUNG Electronics} % Organization
    {S.Korea} % Location
    {Sep. 2013, Mar. 2011 - Oct. 2011} % Date(s)
    {
      \begin{cvitems} % Description(s) of tasks/responsibilities
        \item {Conducted penetration testing on SAMSUNG KNOX, which is solution for enterprise mobile security.}
        \item {Conducted penetration testing on SAMSUNG Smart TV.}
      \end{cvitems}
      %\begin{cvsubentries}
      %  \cvsubentry{}{KNOX(Solution for Enterprise Mobile Security) Penetration Testing}{Sep. 2013}{}
      %  \cvsubentry{}{Smart TV Penetration Testing}{Mar. 2011 - Oct. 2011}{}
      %\end{cvsubentries}
    }

%---------------------------------------------------------
\end{cventries}

%-------------------------------------------------------------------------------
%	SECTION TITLE
%-------------------------------------------------------------------------------
\cvsection{Education}


%-------------------------------------------------------------------------------
%	CONTENT
%-------------------------------------------------------------------------------
\begin{cventries}

%---------------------------------------------------------

  \begin{luacode}
    for k,edu in ipairs(resume['education']) do
      tex.sprint(
        string.format(
          '\\cventry{\%s | \%s}{\%s}{\%s}{\%s - \%s}{}',
          edu['degrees'][1]['degreeName'],
          edu['degrees'][1]['focus'],
          edu['organization'],
          'location',
          edu['start'],
          edu['end']
        )
      )
      tex.sprint('\\vspace{3mm}')
    end
  \end{luacode}

  % \cventry
  %   {B.S. in Computer Science and Engineering} % Degree
  %   {POSTECH(Pohang University of Science and Technology)} % Institution
  %   {Pohang, S.Korea} % Location
  %   {Mar. 2010 - Aug. 2017} % Date(s)
  %   {
  %     \begin{cvitems} % Description(s) bullet points
  %       \item {Got a Chun Shin-Il Scholarship which is given to promising students in CSE Dept.}
  %     \end{cvitems}
  %   }

%---------------------------------------------------------
\end{cventries}

% %-------------------------------------------------------------------------------
%	SECTION TITLE
%-------------------------------------------------------------------------------
\cvsection{Honors \& Awards}


%-------------------------------------------------------------------------------
%	SUBSECTION TITLE
%-------------------------------------------------------------------------------
\cvsubsection{International}


%-------------------------------------------------------------------------------
%	CONTENT
%-------------------------------------------------------------------------------
\begin{cvhonors}

%---------------------------------------------------------
  \cvhonor
    {Finalist} % Award
    {DEFCON 26th CTF Hacking Competition World Final} % Event
    {Las Vegas, U.S.A} % Location
    {2018} % Date(s)

%---------------------------------------------------------
  \cvhonor
    {Finalist} % Award
    {DEFCON 25th CTF Hacking Competition World Final} % Event
    {Las Vegas, U.S.A} % Location
    {2017} % Date(s)

%---------------------------------------------------------
  \cvhonor
    {Finalist} % Award
    {DEFCON 22nd CTF Hacking Competition World Final} % Event
    {Las Vegas, U.S.A} % Location
    {2014} % Date(s)

%---------------------------------------------------------
  \cvhonor
    {Finalist} % Award
    {DEFCON 21st CTF Hacking Competition World Final} % Event
    {Las Vegas, U.S.A} % Location
    {2013} % Date(s)

%---------------------------------------------------------
  \cvhonor
    {Finalist} % Award
    {DEFCON 19th CTF Hacking Competition World Final} % Event
    {Las Vegas, U.S.A} % Location
    {2011} % Date(s)

%---------------------------------------------------------
\end{cvhonors}


%-------------------------------------------------------------------------------
%	SUBSECTION TITLE
%-------------------------------------------------------------------------------
\cvsubsection{Domestic}


%-------------------------------------------------------------------------------
%	CONTENT
%-------------------------------------------------------------------------------
\begin{cvhonors}

%---------------------------------------------------------
  \cvhonor
    {3rd Place} % Award
    {WITHCON Hacking Competition Final} % Event
    {Seoul, S.Korea} % Location
    {2015} % Date(s)

%---------------------------------------------------------
  \cvhonor
    {Silver Prize} % Award
    {KISA HDCON Hacking Competition Final} % Event
    {Seoul, S.Korea} % Location
    {2017} % Date(s)

%---------------------------------------------------------
  \cvhonor
    {Silver Prize} % Award
    {KISA HDCON Hacking Competition Final} % Event
    {Seoul, S.Korea} % Location
    {2013} % Date(s)

%---------------------------------------------------------
\end{cvhonors}

% %-------------------------------------------------------------------------------
%	SECTION TITLE
%-------------------------------------------------------------------------------
\cvsection{Presentation}


%-------------------------------------------------------------------------------
%	CONTENT
%-------------------------------------------------------------------------------
\begin{cventries}

%---------------------------------------------------------
  \cventry
    {Presenter for <Hosting Web Application for Free utilizing GitHub, Netlify and CloudFlare>} % Role
    {DevFest Seoul by Google Developer Group Korea} % Event
    {Seoul, S.Korea} % Location
    {Nov. 2017} % Date(s)
    {
      \begin{cvitems} % Description(s)
        \item {Introduced the history of web technology and the JAM stack which is for the modern web application development.}
        \item {Introduced how to freely host the web application with high performance utilizing global CDN services.}
      \end{cvitems}
    }

%---------------------------------------------------------
  \cventry
    {Presenter for <DEFCON 20th : The way to go to Las Vegas>} % Role
    {6th CodeEngn (Reverse Engineering Conference)} % Event
    {Seoul, S.Korea} % Location
    {Jul. 2012} % Date(s)
    {
      \begin{cvitems} % Description(s)
        \item {Introduced CTF(Capture the Flag) hacking competition and advanced techniques and strategy for CTF}
      \end{cvitems}
    }

%---------------------------------------------------------
\end{cventries}

% %-------------------------------------------------------------------------------
%	SECTION TITLE
%-------------------------------------------------------------------------------
\cvsection{Writing}


%-------------------------------------------------------------------------------
%	CONTENT
%-------------------------------------------------------------------------------
\begin{cventries}

%---------------------------------------------------------
  \cventry
    {Founder \& Writer} % Role
    {A Guide for Developers in Start-up} % Title
    {Facebook Page} % Location
    {Jan. 2015 - PRESENT} % Date(s)
    {
      \begin{cvitems} % Description(s)
        \item {Drafted daily news for developers in Korea about IT technologies, issues about start-up.}
      \end{cvitems}
    }

%---------------------------------------------------------
\end{cventries}

% %-------------------------------------------------------------------------------
%	SECTION TITLE
%-------------------------------------------------------------------------------
\cvsection{Program Committees}


%-------------------------------------------------------------------------------
%	CONTENT
%-------------------------------------------------------------------------------
\begin{cvhonors}

%---------------------------------------------------------
  \cvhonor
    {Problem Writer} % Position
    {2016 CODEGATE Hacking Competition World Final} % Committee
    {S.Korea} % Location
    {2016} % Date(s)

%---------------------------------------------------------
  \cvhonor
    {Organizer \& Co-director} % Position
    {1st POSTECH Hackathon} % Committee
    {S.Korea} % Location
    {2013} % Date(s)

%---------------------------------------------------------
\end{cvhonors}

%-------------------------------------------------------------------------------
%	SECTION TITLE
%-------------------------------------------------------------------------------
\cvsection{Skills}


%-------------------------------------------------------------------------------
%	CONTENT
%-------------------------------------------------------------------------------
\begin{cventries}

%---------------------------------------------------------


  \begin{luacode}
    for k,skill in ipairs(resume['skills']) do
      local skills = skill['skills']
      local items = '\\begin{cvitems}\\item {'
        for j,s in ipairs(skills) do
        if j < #skills then
          items = items .. string.format('\%s, ', s)
        else
          items = items .. string.format('\%s', s)
        end
      end
      items = items .. '}\\end{cvitems}'
      items = string.gsub(items, '\#', '\\\#')

      tex.sprint(
        string.format(
          '\\cventry{\%s}{}{}{}{\%s}',
          skill['skillType'],
          items
        )
      )
      tex.sprint('\\vspace{3mm}')
    end
  \end{luacode}


  % \cventry
  %   {Core Member \& President at 2013} % Affiliation/role
  %   {PoApper (Developers' Network of POSTECH)} % Organization/group
  %   {Pohang, S.Korea} % Location
  %   {Jun. 2010 - Jun. 2017} % Date(s)
  %   {
  %     \begin{cvitems} % Description(s) of experience/contributions/knowledge
  %       \item {Reformed the society focusing on software engineering and building network on and off campus.}
  %       \item {Proposed various marketing and network activities to raise awareness.}
  %     \end{cvitems}
  %   }

%---------------------------------------------------------
\end{cventries}


%-------------------------------------------------------------------------------

% disable pagenumber
\pagenumbering{gobble}

%-------------------------------------------------------------------------------
\end{document}
